%exp18
The two models that used LSTM layers without convolutional layers presented slower learning in comparison with the others, as can be seen in figure \ref{fig:learning}. The validation accuracy values can be consulted in figure \ref{fig:models}.

\noindent
\begin{figure*}[htb!]
\centering\includegraphics[width=0.9\textwidth]{content/epoch_val_categorical_accuracy.png}
\caption{\label{fig:learning}Learning curves of some models (validation accuracy vs. time)}%
\end{figure*}

\noindent
\begin{figure*}[htb!]
\centering\includegraphics[width=1.0\textwidth]{content/models.png}
\caption{\label{fig:models}The bar plot shows the validation accuracy of the considered models. Their training time, in minutes, and the total epochs processed is also indicated. The training was interrupted when no further improvement was observed for 10 consecutive epochs.}%
\end{figure*}


The validation accuracy values of the remaining models were in the 0.35-0.54 range.

% The selected model for further research, identified as ``CM'', achieved an accuracy of 0.54. The models, ``CLD'' and ``CCM'' achieved similar values, but model ``CM'' performed better in the experiment described in section \ref{sec:exprandom}.

To check if the two slower models could give better results if executed for a longer period, they were trained for 10 hours. The model identified as ``LD'', which uses an LSTM layer followed by a fully connected layer, was able to achieve an accuracy of 0.494, while the other, ``LL'', achieved only 0.153.

%stagnation in learning with such short times, combined with low final accuracy suggests that the dataset present some patterns that are easily recognizable, while the rest of the dataset present a very hard classification task.

% \levelC{Limitations and threats to validity}
