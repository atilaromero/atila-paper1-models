This study evaluates some alternative models in the file fragment classification task. The expectation was to identify the most promising models for improvement. But an apparent limit was found on how far these models could be improved.

% Using different combinations of layers and parameters, the expectation was that some models would be discarded and some would be selected as promising alternatives. That could be used to guide future researches and also serve as a reference for comparison between studies. 
The Govdocs1 dataset brought an important basis for comparison to be used between carving solutions. But to achieve easily reproducible results, the models must also be publicly available, a condition not all revised studies fulfill. Being available as Jupyter notebooks at https://github.com/atilaromero/carving-experiments, the results described here should require little effort to be reproduced. Thus the models presented here can be used as a basis of comparison in  future researches.
