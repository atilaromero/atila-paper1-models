
%from pep 2.1, paragrafos 12, 13
Amirani \textit{et al.}  \cite{amirani_new_2008} appear to be the first 
to provide a viable alternative to classical data carving tools using a Neural Network approach. Two previous works were found using Neural Networks with data carving related goals, by Dunham \textit{et al.} \cite{dunham_classifying_2005} and Harris \cite{harris_using_2007}, but the first worked with encrypted files only and the second did not achieve good results.

%from pep 2.1, paragrafos 14
Amirani \textit{et al.}  \cite{amirani_new_2008} used Principal Component Analysis (PCA) as input for a 5 layer feed-forward auto-associative unsupervised neural network to do feature extraction and a 3 layer Multilayer Perceptron (MLP) to perform classification. They used a similar approach in 2013 \cite{amirani_feature-based_2013}.
\sigla{PCA}{Principal Component Analysis}
\sigla{MLP}{Multilayer Perceptron}

%from pep 2.1, paragrafos 16,17,18
Other works also applied Neural Networks to perform data carving using some form of dimensionality reduction, as PCA. Ahmed \textit{et al.} \cite{ahmed_content-based_2010}\cite{ahmed_fast_2011} used byte frequency, 
Penrose \textit{et al.} \cite{penrose_approaches_2013} used compression rate,
and Maslim \textit{et al.} \cite{maslim_distributed_2014} used PCA, as did Amirani \textit{et al.}  \cite{amirani_new_2008}.
%from pep 2.1, paragrafos 15
Xu and Dong \cite{xu_reassembling_2009} used a Neural Network as a cluster reassembling technique for JPEG image fragments.

%from pep 2.1, paragrafos 19
Hiester \cite{hiester_file_2018} apparently was the first to not resort to dimensionality reduction and also the first to utilize an LSTM network to perform file fragment classification. He compared results using three types of neural networks: feedforward, convolutional, and LSTM. The goal was to classify the data type of individual sectors (512 bytes), considering four file types: CSV, XML, JPG and GIF.

% Table \ref{tab:datacarvingstudies} summarizes the  machine learning techniques used in each data carving study.
% \input{content/tables/3.3-table-studies.tex}